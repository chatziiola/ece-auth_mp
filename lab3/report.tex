% Created 2024-06-07 Fri 15:04
% Intended LaTeX compiler: xelatex
\documentclass[11pt]{article}
\input{~/Github/org-to-latex-export/mytemplate.tex}

\usepackage{minted}
\usepackage{polyglossia}
\setmainlanguage[variant=usmax]{english}
\author{Γιώργος Σελιβάνωφ, Λαμπρινός Χατζηιώαννου}
\date{\today}
\title{Εργασία 3η}
\hypersetup{
 pdfauthor={Γιώργος Σελιβάνωφ, Λαμπρινός Χατζηιώαννου},
 pdftitle={Εργασία 3η},
 pdfkeywords={},
 pdfsubject={Εργασία στο μάθημα των MP},
 pdfcreator={Emacs 29.2 (Org mode 9.7-pre)}, 
 pdflang={English}}
\begin{document}

\maketitle
\section{Γενικά}
\label{sec:org1d7ea7e}
Η δομή του προγράμματος παρουσιάζει αρκετές ομοιότητες με εκείνη της
δεύτερης εργασίας. Οι κυριότερες διαφορές είναι η εισαγωγή δεύτερου
timer (\texttt{tim.h}), ο οποίος, με δομή παρόμοια με τον δοθέντα timer
χρησιμοποιείται αντί \texttt{delay} για τον ρυθμό δειγματοληψίας, και η
ανάπτυξη \texttt{drivers} για τα εμπλεκόμενα περιφερειακά.
\section{DHT: Θερμοκρασία και υγρασία}
\label{sec:org25f0554}
Για τον έλεγχο του χρήστη, όπως άλλωστε απαιτούνταν, γράφτηκαν οι
κατάλληλοι drivers (\texttt{dht.c/h}).  Ουσιαστικά, ακολουθώντας τα αρχεία
προδιαγραφών, και μεριμνώντας για τις συνθήκες ελαττωματικών
μετρήσεων, γίνεται η ερμηνεία των δεδομένων που εισέρχονται στο Pin,
και αποθηκεύονται στην επιθυμητή μορφή.
\section{ISRs}
\label{sec:org81a3a92}
\subsection{TouchSensor ISR}
\label{sec:orga41356c}
Ρύθμιση του ISR ούτως ώστε η κλήση του να συνεπάγεται (ανάλογα το
πλήθος των προηγούμενων πατημάτων)
\begin{itemize}
\item την αρχική συμπεριφορά
\item την συμπεριφορά για ζυγό αριθμό "κλήσεων"
\item την συμπεριφορά για μονό αριθμό κλήσεων.
\end{itemize}
\subsection{Temperature (Timer) ISR}
\label{sec:org4ea5fde}
Από την εκφώνηση ζητείται η επανειλλημένη (σε μεταβλητά κατα την
διάρκεια του προγράμματος διαστηματα) μέτρηση θερμοκρασίας (ή και
υγρασίας). Αυτό, το πετύχαμε με την χρήση του δικού μας \emph{δευτερεύοντος}
timer, ο οποίος είχε δυνατότητα να προκαλεί ISR μέχρι και 18
δευτερόλεπτα (που είναι άλλωστε το ανώτατο δυνατό όριο για τις
προδιαγραφές του συστήματος). 
\subsection{Blink ISR}
\label{sec:org5a6ee31}
Ακόμα μία εξεταζόμενη περίπτωση είναι η συμπεριφορά του LED ανάλογα με
τις τιμές της θερμοκρασίας. Ο έλεγχος, λαμβάνει χώρα κατα την διάρκεια
του check message, και ανάλογα με τις υπάρχουσες τιμές επιδρά επί του
\emph{πρωτεύοντος} \texttt{timer} (από τους driver του εργαστηρίου). Στην περίπτωση
που είναι ενεργοποιημένος, η \texttt{blink\_isr} είναι υπεύθυνη για την
περιοδική αλλαγή των καταστάσεων του LED.
\section{Εκτύπωση}
\label{sec:org4c111d3}
Κρατήθηκε παρόμοια μεθοδολογία με ύπαρξη \texttt{msg\_queue}, προς αποφυγή της
χρήσης εκτυπώσεων εντός των ISR. Μια μικρή λεπτομέρεια στο
συγκεκριμένο είναι πως για την "καλύτερη" λειτουργία βάσει των
προδιαγραφών σχηματίστηκε και \texttt{float\_queue}.
\begin{minted}[]{c}
while(1)
{
        if(!checkMessg()) {
                __WFI();
        }	
}
\end{minted}

Ακόμα \textbf{σημαντικό} είναι να τονιστεί πως αποφασίστηκε η εξής "απλοποίηση"
για λόγους μνήμης/ταχύτητας: Χρησιμοποιείται το ίδιο Queue τόσο για
θερμοκρασία, υγρασία αλλά και ρυθμό δειγματοληψίας. Για την διασφάλιση
της κατάλληλης ερμηνείας μεταξύ των, δόθηκε ξεχωριστό value range, στο
καθένα από αυτά μήκους 255, καθώς οι τιμές τους εκ φύσεως δεν μπορούν
να είναι μεγαλύτερες από την μέγιστη αποθηκεύσιμη σε 8 bit. Κατα την
εκτύπωση γίνεται ο κατάλληλος διαχωρισμός/ερμηνεία:

\begin{minted}[]{c}
if(msg < 255) {
        // Temperature logic
        ...
        sprintf(buffer, "Temperature: %f C.\r\n", (double)msg);
} else if (msg < 510) {
        // Humidity
        sprintf(buffer, "Humidity: %f %%.\r\n", (double)(msg-255));
} else {
        // Sensor rate
        sprintf(buffer, "Sensor Rate: %f s .\r\n", (double)(msg-510));
}
\end{minted}
\end{document}
