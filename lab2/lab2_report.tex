% Created 2024-05-27 Mon 23:38
% Intended LaTeX compiler: xelatex
\documentclass[11pt]{article}
\input{~/Github/org-to-latex-export/mytemplate.tex}

\usepackage{polyglossia}
\setmainlanguage[variant=usmax]{english}
\author{Λαμπρινός Χατζηιωάννου, Γιώργος Σελιβάνωφ}
\date{\today}
\title{Εργαστήριο 2 - Αναφορά}
\hypersetup{
 pdfauthor={Λαμπρινός Χατζηιωάννου, Γιώργος Σελιβάνωφ},
 pdftitle={Εργαστήριο 2 - Αναφορά},
 pdfkeywords={},
 pdfsubject={Αναφορά για το Εργαστήριο 2 του μαθήματος Μικροεπεξεργαστές και Περιφερειακά},
 pdfcreator={Emacs 29.2 (Org mode 9.7-pre)}, 
 pdflang={English}}
\begin{document}

\maketitle
\tableofcontents

Για την υλοποίηση της 2ης εργασίας, αποφασίστηκε να μην
χρησιμοποιηθούν \texttt{uart\_prints} μέσα στα interrupts (ή στα callbacks
αυτών), ώστε να διατηρηθεί το execution time τους στο ελάχιστο.
Αντ’αυτού όλες οι ενέργειες που απαιτούν \texttt{uart\_print} εκτελούνται στην
\texttt{main} εκμεταλλευόμενοι \texttt{queues}.
\section{ΑΕΜ μέσω UART}
\label{sec:orgcf79778}
Για την ανάγνωση του ΑΕΜ μέσω UART χρησιμοποιήθηκαν οι συναρτήσεις που
δόθηκαν στα πλαίσια του εργαστηρίου (\texttt{uart.c / uart.h}). Αντί η \texttt{main} να
αναμένει ολόκληρο το input του χρήστη, πλέον χειρίζεται τον κάθε
χαρακτήρα ξεχωριστά, ώστε να μπορεί να εκτελέσει και άλλες ενέργειες
σε κάθε επανάληψη του loop (π.χ. Εκτύπωση της αλλαγής της κατάστασης
του LED). Ως εκ τούτου, το enter αναγνωρίζεται ομοίως με το backspace,
εκτελώντας μια ξεχωριστή συνάρτηση κατά τον εντοπισμό του
(\texttt{checkAEM()}).

\begin{verbatim}
while(1) 
    if (!( checkUart(&buff_index,buff) || checkMessg() ))
      __WFI();
\end{verbatim}
\section{Συμπεριφορά LED ανάλογα με το AEM}
\label{sec:orgd417858}
Για το blink του LED ορίστηκε ένας timer με την χρήση των \texttt{timer.c} και
\texttt{timer.h} που δίνονται στα πλαίσια του εργαστηρίου, ο οποίος απλώς κάνει
toggle το \texttt{GPIO PA\_5} (User LED GPIO των SMT32 board). Κατά τον έλεγχο
του ΑΕΜ με την \texttt{checkAEM()}, ο συγκεκριμένος timer ενεργοποιείται ή
απενεργοποιείται αναλόγως με τον εάν το ΑΕΜ είναι μονός ή ζυγός
αριθμός αντίστοιχα.

Να σημειωθεί ότι παρότι στην εκφώνηση αναφέρεται πως τα παραπάνω
πρέπει να γίνουν με την χρήση \texttt{ISR}, δεν βρήκαμε πως η χρήση ISR θα ήταν
εφικτή/ενδεδειγμένη στο συγκεκριμένο στάδιο. Το input από τον χρήστη
γίνεται στο στάδιο της UART, στο οποίο υπάρχει ήδη \texttt{ISR} μέσα στην οποία
δεν θα ήταν σκόπιμο να ενσωματωθεί όλος αυτός ο έλεγχος.
\section{Πάτημα του διακόπτη}
\label{sec:org31a7316}
Τα ζητούμενα e) και f) ουσιαστικά ζητάνε το πάτημα του διακόπτη να
κάνει toggle το LED και να εκτυπώνει τον coutner των πατημάτων.
Χρησιμοποιώντας της συναρτήσεις που δίνονται στα \texttt{gpio.c} και \texttt{gpio.h},
ορίσαμε μια \textbf{interrupt callback} η οποία κάνει toggle το LED και
προσθέτει το κατάλληλο μήνυμα στη \hyperref[sec:org3847045]{message queue}, ώστε να εκτυπωθεί ο
counter από την main.
\section{Message Queue}
\label{sec:org3847045}
Για την εκτύπωση μηνυμάτων μέσω uart\textsubscript{print} από τις διάφορες ISR, έγινε
χρήση ενός message queue. Οι \texttt{ISR} προσθέτουν integers στο queue, οι
οποίοι αντιστοιχίζονται στις αντίστοιχες ενέργειες:
\begin{enumerate}
\item Εκτύπωση μηνύματος για απενεργοποίηση του LED
\item Εκτύπωση μηνύματος για ενεργοποίηση του LED
\item Αύξηση κατά 1 και εκτύπωση του counter
\end{enumerate}
Με αυτόν τον τρόπο, δεν επιβαρύνονται τα interrupts από την uart\textsubscript{print}, διατηρώντας τον χρόνο εκτέλεσής τους χαμηλό.

\begin{verbatim}
if (queue_dequeue(&msg_queue, &lastState)) {
  switch (lastState) {
  case 0:
    sprintf(buffer, "Led is off.\r\n"); break;
  case 1:
    sprintf(buffer, "Led is on.\r\n"); break;
  case 2:
    sprintf(buffer, "Keypresses pressed %d .\r\n", ++switchPresses); break;
  }
\end{verbatim}
\end{document}
