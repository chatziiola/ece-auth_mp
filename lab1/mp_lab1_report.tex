% Created 2024-04-11 Thu 12:31
% Intended LaTeX compiler: xelatex
\documentclass[11pt]{article}
\input{~/Github/org-to-latex-export/mytemplate.tex}

\usepackage{polyglossia}
\setmainlanguage[variant=usmax]{english}
\author{Λαμπρινός Χατζηιωάννου, Γιώργος Σελιβάνωφ}
\date{\today}
\title{Εργαστήριο 1}
\hypersetup{
 pdfauthor={Λαμπρινός Χατζηιωάννου, Γιώργος Σελιβάνωφ},
 pdftitle={Εργαστήριο 1},
 pdfkeywords={},
 pdfsubject={description},
 pdfcreator={Emacs 29.2 (Org mode 9.7-pre)}, 
 pdflang={English}}
\begin{document}

\maketitle
\tableofcontents

\section{Εισαγωγή}
\label{sec:org174c111}

\section{Hashing Function}
\label{sec:org298e7a2}
\begin{verbatim}
int hashfunc(char inputString[]) {
#if defined(DEBUG)
  printf("\tStarting hashfunc with %s\n", inputString);
#endif
  // Verified proper length
  int values[] = {10, 42, 12, 21, 7, 5,  67, 48, 69, 2, 36, 3,  19,
                  1,  14, 51, 71, 8, 26, 54, 75, 15, 6, 59, 13, 25};
  int hash = 0;

  for (int ind = 0; inputString[ind]!='\0'; ind++) {
    // ASCII:
    // - Caps: 64-91 (non-inclusive ranges here)
    // - Lower: 96 - 123 (non-inclusive)
    // - Digits: 47-58 (non-inclusive)
#if defined(DEBUG)
    printf("\tAt char \'%c\', ascii %d\n", inputString[ind], inputString[ind]);
#endif
    if (inputString[ind] > 64 && inputString[ind] < 91) {
      // Meaning caps
      hash += values[inputString[ind] - 65];
#if defined(DEBUG)
      printf("\t\tAdding %d (%d)\n",
             values[inputString[ind] - 65],
             inputString[ind] - 65);
#endif
    } else if (inputString[ind] > 96 && inputString[ind] < 123) {
      // Meaning lowercase
      hash -= values[inputString[ind] - 97];
#if defined(DEBUG)
      printf("\t\tSubtracting %d (%d)\n",
             values[inputString[ind] - 97],
             inputString[ind] - 97);
#endif
    } else if (inputString[ind] > 47 && inputString[ind] < 58) {
      // Meaning integer
      hash += inputString[ind] - 48;
#if defined(DEBUG)
      printf("\t\tAdding %d (int)\n",
             inputString[ind] - 48);
#endif
    }
  }
  return hash;
}

\end{verbatim}
\section{Sum of Natural Numbers}
\label{sec:org614a35a}
\begin{verbatim}
int sum_of_natural_numbers(int n)
{
  int result = 0; if(n > 0)
                    {
                      result = n + sum_of_natural_numbers(n-1);
                    }
  return result;
}
\end{verbatim}

\$
\section{}
\label{sec:orgfb161cd}
\end{document}
